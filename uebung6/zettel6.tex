\documentclass{article}
\usepackage[a4paper, margin=0.8in]{geometry}
\usepackage[utf8]{inputenc}                     % format
\usepackage[english]{babel}                     % language

\usepackage{booktabs}
\usepackage[scale=2]{ccicons}
\usepackage{blindtext}
\usepackage{minted}
% CUSTOM PACKAGES
\usepackage{mathtools}
\usepackage{array}
\usepackage{mathtools}
\usepackage{amssymb}
\usepackage{amsthm}
\usepackage{hyperref}
\PassOptionsToPackage{hyphens}{url} % linebreaks in URLs

\newcolumntype{P}[1]{>{\centering\arraybackslash}p{#1}}
\newcounter{colnum}
\newcommand\colnumber{\stepcounter{colnum}\arabic{colnum}}
\renewcommand{\UrlFont}{\scriptsize}
\usepackage{graphicx}
\usepackage{listings}
\renewcommand{\arraystretch}{1.5}

\title{Datenbanken: Übungsblatt 5}
\author{$\bigl($Florian Rauls, \textbf{Christian Heusel}, Philipp Göldner$\bigr)$ (Gruppe 1)}
\date{}


\begin{document}

    \vspace*{-5em}
    {\let\newpage\relax\maketitle}

    \begin{center}
        \begin{tabular}{*3{|c}||P{1.5cm}|}
            \hline
            Aufgabe \colnumber{} & Aufgabe \colnumber{} & Aufgabe \colnumber{} & $\sum$ \\[5pt]
            \hline
             & & & \\[5pt]
            \hline
        \end{tabular}
    \end{center}

    \section*{Aufgabe 1}
    \begin{enumerate}
        \item
            \begin{minted}{SQL}
WITH RECURSIVE Fliesst(Fluss, Muendung) AS
(
    SELECT Fluss, Muendung
    FROM Wasserlauf W
    WHERE fluss = 'Schorte'

    UNION ALL

    SELECT f.Fluss, w.Muendung
    FROM Fliesst f, Wasserlauf W
    WHERE f.muendung = w.fluss
)
SELECT * FROM Fliesst
            \end{minted}
            \begin{tabular}{|c|c|}
                \texttt{fluss} & \texttt{Muendung} \\
                \toprule
                \texttt{Schorte} & \texttt{Ilm} \\
                \texttt{Schorte} & \texttt{Saale} \\
                \texttt{Schorte} & \texttt{Elbe} \\
                \texttt{Schorte} & \texttt{Nordsee} \\
            \end{tabular}
        \item
            \begin{minted}{SQL}
WITH RECURSIVE Fliesst(Fluss, Muendung, laenge) AS
(
    SELECT Fluss, Muendung, laenge
    FROM Wasserlauf W
    UNION ALL
    SELECT f.Fluss, w.Muendung, f.laenge+w.laenge
    FROM Fliesst f, Wasserlauf W
    WHERE f.muendung = w.fluss
)
SELECT fluss AS maxabstand_schwarzes_meer FROM Fliesst
WHERE muendung = 'Schwarzes Meer'
    AND laenge >= (SELECT MAX(laenge) FROM fliesst)
            \end{minted}
            \begin{tabular}{|c|}
                \texttt{maxabstand\_schwarzes\_meer} \\
                \toprule
                \texttt{Inn} \\
            \end{tabular}
    \end{enumerate}

\section*{Aufgabe 3}
    \begin{enumerate}
        \item
            \begin{enumerate}
                \item[i)] Gib den Namen aller Konzerte zurück, auf denen 'Data Control' gespielt wurde.
                \item[ii)] $\{\texttt{N} | \texttt{CONCERT(\_, \_, N, \_, \_, \_, S)} \land \texttt{SONGS(S, \_, SN)} \land \texttt{SN='Data Control'}\}$
            \end{enumerate}
        \item
            \pagebreak
            \begin{enumerate}
                \item[i)] Gib den Namen aller Bands zurück, in deren Lineup Geschwister spielen.
                \item[ii)]
                    \begin{equation}
                    \begin{align}
                        {
                            \{\texttt{<a.name>} | \texttt{ar0},\texttt{ ar1},\texttt{ ar2} \in \texttt{ARTIST\_RELATION } \land & \texttt{ ar0.type = 'sibling'}  \\
                                                                                                                          \land & \texttt{ ar1.type = 'member of band'} \\
                                                                                                                          \land & \texttt{ ar0.artist2 = ar1.artist0} \\
                                                                                                                          \land & \texttt{ ar2.artist1 = ar1.artist1} \\
                                                                                                                          \land & \texttt{ ar2.artist0 = ar0.artist1} \\
                                                                                                                          \land & \texttt{ ar2.type = 'member of band'} \\
                                                                                                                          \land & \texttt{ a = 'Group'} \\
                                                                                                                          \land & \texttt{ a.id = ar1.artist1} \}
                        }
                    \end{align}
                \end{equation}
            \end{enumerate}
    \end{enumerate}
\end{document}
